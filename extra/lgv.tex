\documentclass[10pt,a4paper]{article}
\usepackage[utf8]{inputenc}
\usepackage{amsmath}
\usepackage{amsfonts}
\usepackage{amssymb}
\usepackage{xcolor}


\newcommand{\ppi}{\frac{\partial}{\partial p_i}}
\newcommand{\ppj}{\frac{\partial}{\partial p_j}}
\newcommand{\ppl}{\frac{\partial}{\partial p_l}}
\newcommand{\Kpara}{\kappa_{\|}}
\newcommand{\Kperp}{\kappa_{\perp}}
\newcommand{\Ppara}{\hat{P}^{\|}}
\newcommand{\Pperp}{\hat{P}^{\perp}}


\usepackage{amsmath}

\DeclareFontFamily{OMS}{oasy}{\skewchar\font48 }
\DeclareFontShape{OMS}{oasy}{m}{n}{%
         <-5.5> oasy5     <5.5-6.5> oasy6
      <6.5-7.5> oasy7     <7.5-8.5> oasy8
      <8.5-9.5> oasy9     <9.5->  oasy10
      }{}
\DeclareFontShape{OMS}{oasy}{b}{n}{%
       <-6> oabsy5
      <6-8> oabsy7
      <8->  oabsy10
      }{}
\DeclareSymbolFont{oasy}{OMS}{oasy}{m}{n}
\SetSymbolFont{oasy}{bold}{OMS}{oasy}{b}{n}

\DeclareMathSymbol{\smallleftarrow}     {\mathrel}{oasy}{"20}
\DeclareMathSymbol{\smallrightarrow}    {\mathrel}{oasy}{"21}
\DeclareMathSymbol{\smallleftrightarrow}{\mathrel}{oasy}{"24}

\newcommand{\tensor}[1]{\overset{\scriptscriptstyle\smallleftrightarrow}{#1}}

\begin{document}
\section{Reduce Fokker-Plank equation to Langevin dynamics of test particles.}
The small-momenta transfer approximation reduces the Linear Boltzmann equation to a Fokker-Plank equation,
\begin{eqnarray}
\frac{\partial f}{\partial t} + v\cdot\frac{\partial f}{\partial x}
= \ppi \left(A(p^2) p_i + \frac{1}{2}\ppj B_{ij}\right)f 
\end{eqnarray}
$A$ is drag term. The momentum diffusion term can be can be decomposed into,
\begin{eqnarray}
B_{ij} = \Kpara \frac{p_i p_j}{p^2} + \Kperp \left(\delta_{ij} - \frac{p_i p_j}{p^2}\right)
\end{eqnarray}
To guarantee the system has a thermalized solution, $A$, $\Kpara$ and $\Kperp$ are not independent.
Given a static and homogeneous medium at equilibrium with temperature $T$, $f = N\exp\left(-\beta E\right)$, the equation reduces to
\begin{eqnarray}
0 &=& \ppi(\phi p_i f)\\
\phi &=& A - \frac{\Kpara}{2TE} + \frac{\partial \Kpara}{\partial p^2} + \frac{\Kpara-\Kperp}{p^2}.
\end{eqnarray}
The Einstein relation $\phi = 0$ guarantees the exist of a equilibrium solution,
\begin{eqnarray}
A = \frac{\Kpara}{2TE} - \frac{\partial \Kpara}{\partial p^2} - \frac{\Kpara-\Kperp}
\Delta \vec{x}_i &=& \frac{p^2}.
\end{eqnarray}

The Langevin equation governs the evolution of an ensemble of test particles whose weighted distribution is the solution of the Fokker-Plank equation.
\begin{eqnarray}
f(t,x,p) \approx \sum_{i} w_i \delta^3(x-x_i(t)) \delta^3(p-p_i(t))
\end{eqnarray}
The Langevin equation in the post-point discretization scheme is,
\begin{eqnarray}{\vec{p}_i}{E} \Delta t	\\
\Delta \vec{p}_i &=& -\Gamma \vec{p}_i \Delta t + \sqrt{\tensor{B}(p+\Delta p) \Delta t  }\vec{\xi}
\end{eqnarray}
$\Gamma$ is the Langevin drag term.
The relation between $A$ and $\Gamma$ in the post-point scheme is,
\begin{eqnarray}
p_j \Gamma  = p_jA + \left(\sqrt{\Kpara}\Ppara_{lk} + \sqrt{\Kperp}\Pperp_{lk}\right) \ppl \left( \sqrt{\Kpara}\Ppara_{kj} + \sqrt{\Kperp}\Pperp_{kh} \right).
\end{eqnarray}
This finally reduces to,
\begin{eqnarray}
\Gamma &=& A + \frac{\partial \Kpara}{p^2} + \frac{2\sqrt{\Kpara\Kperp} - 2\Kperp}{p^2} \\
 &=& \frac{\Kpara}{2TE} - \frac{1}{p^2}\left( \sqrt{\Kpara} - \sqrt{\Kperp} \right)^2.
\end{eqnarray}




\end{document}
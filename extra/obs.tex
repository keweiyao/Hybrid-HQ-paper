\documentclass[10pt,a4paper]{article}
\usepackage[utf8]{inputenc}
\usepackage{amsmath}
\usepackage{amsfonts}
\usepackage{amssymb}

\newcommand{\Raa}{R_{AA}}

\begin{document}
\section{Initial production of heavy quark}
Due to the large mass of heavy quark, its production can be described by perturbative QCD calculation.
The structure of the pQCD calculation has the form,
\begin{eqnarray}
	\frac{d\sigma_{A + B \rightarrow H + X}}{p_T d{p_T} dy} = \int d x_a d x_b x_a x_b d x_H x_H f_{A}(x_a, M_f)f_{B}(x_a, M_f)\frac{d\sigma_{a + b \rightarrow Q + X}}{p_T d{p_T} dy} f_{Q \rightarrow H}(x_H)
\end{eqnarray}

\section{Calculation of $\Raa$}
$\Raa$ is defined to be the ratio of differential cross-sections in AA collision to the spectra in pp collision scaled by the average number of binary collision for a given centrality.
The differential cross-sections are usually restricted to certain rapidity cuts.
Take the D-meson $\Raa$ as an example,
\begin{eqnarray}
\Raa = \frac{\left. \left\langle \frac{d\sigma_{AA \rightarrow D}}{dp_T} \right|_{|y|<y_0} \right\rangle}{\left. \langle T_{AA}\rangle \frac{d\sigma_{pp \rightarrow D}}{dp_T}\right|_{|y|<y_0}}
\end{eqnarray}
The average is taken over all events that fulfilling the centrality selection criterion.

For experimental measurements, this definition is straight forward. 
The differential cross-sections in the numerator and the denominator can be measured and $\langle T_{AA}\rangle$ is a model-dependent evaluation of nuclear overlapping function.
For transport calculations, however, some complications arise.
In the linear Boltzmann transport simulation, there are four key ingredients,
\begin{itemize}
\item An initial condition model for both hard process and bulk physics. 
It calculates nuclear overlapping function $T_{AA}$, initial spectrum of charm quark production (per binary collision) and hydrodynamic initial condition for soft particles. 
It also provides the base line in the denominator.
\item A hydrodynamic evolution of the bulk.
\item A transport model for energy loss in the medium, which conserves the  total number of charm quarks and only changes the shape of the spectra.
\item A hadronization routine plus hadronic rescattering.
\end{itemize} .
Express the $\Raa$ with there quantities,
\begin{eqnarray}
\Raa &=& \frac{\left\langle \left.\frac{T_{AA}}{\langle T_{AA}\rangle} \frac{d\sigma_{nn \rightarrow D}}{dp_T} \right|_{|y|<y_0} \right\rangle}{\left.  \frac{\sigma_{pp \rightarrow D}}{dp_T}\right|_{|y|<y_0}} \\
&=& \frac{\left\langle \left.\sigma_{nn \rightarrow D}\right|_{|y|<y_0} \left.\frac{d N_D}{dp_T} \right|_{|y|<y_0} \right\rangle_{w=T_{AA}}}{\left.  \frac{d\sigma_{pp \rightarrow D}}{dp_T}\right|_{|y|<y_0}}.
\end{eqnarray}
$d\sigma_{nn \rightarrow D}/dp_T$ is the D meson production within the rapidity cuts per binary collision.
$d N_D/dp_T$ is the normalized distribution of D meson in the final states which can be obtained directly from the transport simulation.
The average over events is then weighted by the nuclear overlapping function.
The only missing piece that connects our calculation with the observable is the actually yield per binary collision of D meson for each event $\sigma_{nn\rightarrow D}$.
We relate this quantity to the initial charm production cross-section,
\begin{eqnarray}
\left.\sigma_{nn\rightarrow D}\right|_{|y|<y_0} = \left.\sigma_{nn\rightarrow c} \right|_{|y|<y_{max}} \frac{\sum_{|y|<y_0}w_D}{\sum_{|y|<y_{max}}w_c} =  \left.\sigma_{nn\rightarrow c} \right|_{|y|<y_{max}} r_{D/c}.
\end{eqnarray}
This simply represents that the ratio between the total yield of D meson within $|y|<y_0$ to the yield of initial charm quark within $|y| < y_{max}$ equals to the ratio of the respective number of D meson and charm quark in a simulation.
With $\sigma_{nn\rightarrow c}$ calculated from pQCD and the number ratio $r_{D/c}$ from event-by-event simulation, $\Raa$ finally relates to model calculations by,
\begin{eqnarray}
\Raa = \frac{\left\langle \left.\sigma_{nn\rightarrow c} \right|_{|y|<y_{max}} r_{D/c} \left.\frac{d N_D}{dp_T} \right|_{|y|<y_0} \right\rangle_{w=T_{AA}}}{\left.  \frac{d\sigma_{pp \rightarrow D}}{dp_T}\right|_{|y|<y_0}}.
\end{eqnarray}

\section{Calculation of $v_n\{2\}$ and $v_2\{EP\}$}
The calculation of heavy flavor $v_2$ is easier since it measures the amount of anisotrpic and is insensitive to the total yield of heavy flavor hadrons.
Cumulant method correlates a heavy meson with a light particle to estimate $v_n$.
\begin{eqnarray}
v_n\{2\} &=& \frac{d_n\{2\}}{\sqrt{c_n\{2\}} } \\
d_n\{2\} &=& \left.\left\langle \frac{\Re\{pQ^*\}}{mM} \right\rangle\right|_{w = mM} \\
c_n\{2\} &=& \left.\left\langle \frac{|Q|^2-M}{M(M-1)} \right\rangle\right|_{w = M(M-1)}
\end{eqnarray}
\end{document}
\documentclass[10pt,a4paper]{article}
\usepackage[utf8]{inputenc}
\usepackage{amsmath}
\usepackage{amsfonts}
\usepackage{amssymb}

\title{Use rate equation and oversampling to solve Boltzmann equation}
\begin{document}
\section{Target equation}
\begin{eqnarray}
\frac{\partial f_\Upsilon}{\partial t} - \vec{v}\cdot\frac{\partial f_\Upsilon}{\partial x}
= \frac{1}{2E_1}\int d\Gamma_2 d\Gamma_3 d\Gamma_3 |M|^2 \left( f_b(p_3)f_{\bar{b}}(p_4)- f_\Upsilon(p_1)f_0(p_2)\right)  (2\pi)^4  \delta^4(p_1+p_2-p_3-p_4) 
\end{eqnarray}
The evolution of $f_{b, \bar{b}}$ is taken care of by diffusion / LBT dynamics.

\section{Solve by rate equation}
Particles in solving the rate equation may not represent a physical particle, but just a sample point from the distribution $f(p)$.

\begin{itemize}
\item Dissociation rate (probability / unit time):
\begin{eqnarray}
\frac{d P}{d t} = \frac{1}{2E_1}\int d\Gamma_2 d\Gamma_3 d\Gamma_3 |M|^2 f_0(p_2)  (2\pi)^4  \delta^4(p_1+p_2-p_3-p_4) 
\end{eqnarray}
For small enough time-step $\Delta t$, the dissociation probability is $\Delta t \frac{dP}{dt}$. Xiaojun showed last time the dissociation implementation is working as expected.

\item Recombination rates for a $b$ quark (probability / unit time):
\begin{eqnarray}
\frac{d P}{d t} &=& \frac{1}{2 E_3}\int d\Gamma_1 d\Gamma_2 d\Gamma_4 |M|^2 f_{\bar{b}}(p_4) (2\pi)^4  \delta^4(1+2-3-4) \\
&=& \int d p_4^3 f_{\bar{b}}(p_4) [v_{rel}  \sigma_{b\bar{b}\rightarrow \Upsilon}]_{\textrm{calculate in the same frame}}
\end{eqnarray}
The RHS has dimension $[E]^3[L]^2 = [T]^{-1}$ and is interpret as the probability per unit time.

But the integration of $p_4$ is not possible since no analytic form of $f_{\bar{b}}(p_4)$ is available. 
Instead, it is estimated by the surrounding $\bar{b}$ quarks' current status.
The measure on $p_4$ is...
\begin{eqnarray}
\int dp_4^3f_{\bar{b}}(p_4) F(p_4)  \approx \frac{\sum_{x_i \in \Delta V} F(p_i)}{\Delta V}
\end{eqnarray}
$\Delta V$ is some test volume independent of each scattering process in the sum.

Therefore, the rates for recombination process is approximated by,
\begin{eqnarray}
\frac{d P}{d t} &=& \sum_{x_i \in \Delta V}\frac{ [v_{rel}  \sigma_{b\bar{b}\rightarrow \Upsilon}]_{p_4 = p_i} }{\Delta V}
\end{eqnarray}
So, instead of using $\Delta V = [\textrm{relative distance}]^3$ for each term in the scattering, it is the volume for pair searching. 
The probability for of each pair scattering is,
\begin{eqnarray}
p_i = \frac{ dt [v_{rel}  \sigma_{b\bar{b}\rightarrow \Upsilon}]_{p_4 = p_i} }{\Delta V}
\end{eqnarray}
The total recombination probability for this $b$ quark is 
\begin{eqnarray}
P = \sum_{x_i \in \Delta V} p_i
\end{eqnarray}
This approximations requires that the total probability $P$  to be much smaller than $1$.
\end{itemize}

\section{Oversampling and initial correlation}
In the original target equation, if we increase the number of $b$ and $\bar{b}$ and $\Upsilon$ by a factor of $N$ (for example to increase the statistics), to make the equation unchanged, we only need to decrease the recombination cross-section by a factor of $N$ (the decay process is not affected since it is linear).
\begin{eqnarray}
p_i = \frac{ dt [v_{rel}  \sigma_{b\bar{b}\rightarrow \Upsilon}]_{p_4 = p_i} }{N\Delta V}
\end{eqnarray}


\end{document}
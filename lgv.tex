\documentclass[10pt,a4paper]{article}
\usepackage[utf8]{inputenc}
\usepackage{amsmath}
\usepackage{amsfonts}
\usepackage{amssymb}

\newcommand{\ppi}{\frac{\partial}{\partial p_i}}
\newcommand{\ppj}{\frac{\partial}{\partial p_j}}
\newcommand{\ppl}{\frac{\partial}{\partial p_l}}
\newcommand{\Kpara}{\kappa_{\|}}
\newcommand{\Kperp}{\kappa_{\perp}}
\newcommand{\Ppara}{\hat{P}^{\|}}
\newcommand{\Pperp}{\hat{P}^{\perp}}
\begin{document}
\section{Fokker-Plank with anisotropic momentum diffusion.}
The small-momenta transfer approximation reduces the Linear Boltzmann equation to a Fokker-Plank equation,
\begin{eqnarray}
\frac{\partial f}{\partial t} + v\cdot\frac{\partial f}{\partial x}
= \ppi \left(A(p^2) p_i + \frac{1}{2}\ppj B_{ij}\right)f 
\end{eqnarray}
$A(p^2)$ is the momentum drag term. 
By the symmetry of in medium in equilibrium, the momentum diffusion term can be can be decomposed into,
\begin{eqnarray}
B_{ij} = \Kpara \frac{p_i p_j}{p^2} + \Kperp \left(\delta_{ij} - \frac{p_i p_j}{p^2}\right)
\end{eqnarray} 
Where the diffusion coefficient parallel and transverse to the direction of motion $\Kpara$ and $\Kperp$ are functions of $p^2$.

To guarantee the system thermalize at infinite time, $A$, $\Kpara$ and $\Kperp$ are not independent.
Given a static and homogeneous medium at equilibrium with temperature $T$, $f = N\exp\left(-\beta E\right)$, the equation becomes,
\begin{eqnarray}
0 &=& \ppi(\phi p_i f)\\
\phi &=& A - \frac{\Kpara}{TE} + \frac{\partial \Kpara}{\partial p^2} + \frac{\Kpara-\Kperp}{p^2}.
\end{eqnarray}
The Einstein relation $\phi = 0$ guarantees the exist of a equilibrium solution.
Therefore the drag coefficient is related to momentum diffusion via
\begin{eqnarray}
A = \frac{\Kpara}{TE} - \frac{\partial \Kpara}{\partial p^2} - \frac{\Kpara-\Kperp}{p^2}.
\end{eqnarray}

\section{Langevin simulation of the anisotropic Fokker-Plank equation}
The Langevin equation governs the evolution of an ensemble of particles whose distribution is the solution of the Fokker-Plank equation.
The Langevin equation in the post-point discretization scheme is,
\begin{eqnarray}
\Delta x_i &=& \frac{p_i}{E} \Delta t	\\
\Delta p_i &=& -\Gamma p_i \Delta t + \sqrt{\kappa_i(p+\Delta_p)\Delta t} \xi_i
\end{eqnarray}
Where $\Gamma(p^2)$ is the Langevin drag term which may not be the same as $A(p^2)$. 
$\kappa_i$ is the momentum diffusion in that direction. 
It can be projected from $\Kpara$ and $\Kperp$.
The relation between $A$ and $\Gamma$ in the post-point scheme is,
\begin{eqnarray}
p_j \Gamma  = p_jA + \left(\sqrt{\Kpara}\Ppara_{lk} + \sqrt{\Kperp}\Pperp_{lk}\right) \ppl \left( \sqrt{\Kpara}\Ppara_{kj} + \sqrt{\Kperp}\Pperp_{kh} \right).
\end{eqnarray}
This finally reduces to,
\begin{eqnarray}
\Gamma &=& A + \frac{\partial \Kpara}{p^2} + \frac{2\sqrt{\Kpara\Kperp} - 2\Kperp}{p^2}
\end{eqnarray}
And plug in the Einstein relation, the drag coefficient to be used in Langevin simulation is related to momentum diffusion coefficients via,
\begin{eqnarray}
\Gamma &=& \frac{\Kpara}{TE} - \frac{1}{p^2}\left( \sqrt{\Kpara} - \sqrt{\Kperp} \right)^2.
\end{eqnarray}
Discussions of this result,
\begin{itemize}
\item This results in the isotropic case $\Kpara=\Kperp=\kappa$ gives back to $\Gamma = \kappa/TE$.
\item $\Kpara$ and $\Kperp$ from LOpQCD calculations are different, but $\Kpara=\Kperp=\kappa$ is always fulfilled in the limit $p^2 \rightarrow 0$, since there is not a special spatial direction.
\item LOpQCD calculation also violates the Einstein relation. 
This does not prevent the linear Boltzmann equation using the pQCD cross-section from approaching thermal equilibrium. 
However, when comparing to Langevin/Fokker plank dynamics using the same underlying pQCD process, there is definitely an ambiguity that only two of the three quantities from  $A, \Kpara, \Kperp$ from pQCD calculation are need as input, with the rest one (usually the drag coefficient) fixed by the Einstein relation. 
\end{itemize}

\end{document}